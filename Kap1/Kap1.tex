\chapter{Marco Referencial}
\section{Introducción}

El Internet de las cosas (IoT) es considerado como la cúspide en el desarrollo de las tecnologías de la
información y comunicación por lo que su popularidad tanto en la industria como en investigación
son abrumadoras. En los últimos años ha sido convertido en uno de los temas de investigación
más requeridos y al mismo tiempo uno de los más incomprendidos. Existen diversas definiciones
de lo que es y lo que trata, algunas de ellas más relacionadas a posibles aplicaciones y otras a la
tecnología que se encuentra detrás\cite{atzori2017understanding}.\\

Parte de la confusión generada alrededor del IoT lleva principalmente a dos resultados. El primero afecta al
aprendizaje, pues las definiciones establecidas en muchos trabajos proporcionan una idea difusa
y muchas veces incompletas de lo que es el IoT. La segunda se encuentra sobre la aplicación de
esta tecnología dado que resulta difícil para un estudiante encontrarse con definiciones diferentes
y aparentemente contradictorias.\\

En el caso de los estudiantes universitarios de ingeniería mecatrónica que deben utilizar esta
tecnología en sus proyectos se encuentran con la ausencia de una definición clara que dificulta
su investigación y en segundo lugar al propio desarrollo del proyecto. En la asignatura ’Diseño
superior de ingeniería’ en la Universidad Católica Boliviana ”San Pablo”suelen existir proyectos
que contemplan de alguna manera al IoT. La Caja de Skinner IoT en el semestre 1-2019 y
el Invernadero en 2-2019 son ejemplos de proyectos recientes que utilizaron esta tecnología.
Ambos proyectos se encontraron con el problema de dónde y cómo implementarla. Esto en
parte es porque no existe formación previa en relacionada al Internet de las cosas y segundo al
desconocimiento de ecosistemas para el desarrollo de aplicaciones IoT o a la falta de acceso a
dichos ecosistemas.\\

Con el desarrollo del sistema se espera ayudar a que los estudiantes desarrollen sus proyectos de
la mejor manera posible. Del mismo modo, se busca proporcionar las herramientas necesarias
para realizar proyectos tan variados como brazos robóticos tele-operados, sistemas de monitorio
en tiempo real o sistemas de adquisición de datos\cite{alcaraz2014internet}.\\
La enseñanza sobre el Internet de las cosas en muchos lugares del mundo están siendo impulsados
como un objetivo nacional para el futuro. Países en vías de desarrollo como la India están
concentrando sus esfuerzos para convertirse en potencias en el futuro a mediano plazo. Para
ello se llevan a cabo estrategias de desarrollo a nivel educativo, tecnológico e industrial. El IoT
se encuentra en el epicentro de esta revolución, pues sus aplicaciones son innumerables. La
forma de aprendizaje por la que apuestan es el aprendizaje activo, el cual involucra el desarrollo
de proyectos. El vector de aprendizaje que exponen Raikar, Desai y Nragund se encuentra
compuesto por ocho niveles divididos en tres etapas. La primera etapa esta enfocada en la programación, la segunda etapa en el diseño web y la tercera en el desarrollo de aplicaciones. De
este modo, al concluir el programa el estudiante tendrá las competencias necesarias para llevar
a cabo proyectos de principio a fin\cite{raikar2016active}. El modelo propuesto no dista mucho del empleado en
la UCB sin embargo las herramientas puestas a disposición del estudiante se encuentran mejor
definidas empleando la relación herramienta-problema. Por ejemplo, el uso del \textit{framework}
Django para el desarrollo de la interfaz web de tal forma que el estudiante tiene un punto de
partida claro.

% En la introducci\'{o}n, el autor presenta y se\~{n}ala la importancia, el origen (los antecedentes te\'{o}ricos y pr\'{a}cticos), los objetivos, los alcances, las limitaciones, la metodolog\'{\i}a empleada, el significado que el estudio tiene en el avance del campo respectivo y su aplicaci\'{o}n en el \'{a}rea investigada. No debe confundirse con el resumen y se recomienda que la introducci\'{o}n tenga una extensi\'{o}n de m\'{\i}nimo 2 p\'{a}ginas y m\'{a}ximo de 4 p\'{a}ginas.\\

% La presente plantilla maneja una familia de fuentes utilizada generalmente en LaTeX, conocida como Computer Modern, espec\'{\i}ficamente LMRomanM para el texto de los p\'{a}rrafos y CMU Sans Serif para los t\'{\i}tulos y subt\'{\i}tulos. Sin embargo, es posible sugerir otras fuentes tales como Garomond, Calibri, Cambria, Arial o Times New Roman, que por claridad y forma, son adecuadas para la edici\'{o}n de textos acad\'{e}micos.\\

% La presente plantilla tiene en cuenta aspectos importantes de la Norma T\'{e}cnica Colombiana - NTC 1486, con el fin que sea usada para la presentaci\'{o}n final de las tesis de maestr\'{\i}a y doctorado y especializaciones y especialidades en el \'{a}rea de la salud, desarrolladas en la Universidad Nacional de Colombia.\\

% Las m\'{a}rgenes, numeraci\'{o}n, tama\~{n}o de las fuentes y dem\'{a}s aspectos de formato, deben ser conservada de acuerdo con esta plantilla, la cual esta dise\~{n}ada para imprimir por lado y lado en hojas tama\~{n}o carta. Se sugiere que los encabezados cambien seg\'{u}n la secci\'{o}n del documento (para lo cual esta plantilla esta construida por secciones).\\

% Si se requiere ampliar la informaci\'{o}n sobre normas adicionales para la escritura se puede consultar la norma NTC 1486 en la Base de datos del ICONTEC (Normas T\'{e}cnicas Colombianas) disponible en el portal del SINAB de la Universidad Nacional de Colombia\footnote{ver: www.sinab.unal.edu.co}, en la secci\'{o}n "Recursos bibliogr\'{a}ficos" opci\'{o}n "Bases de datos".  Este portal tambi\'{e}n brinda la posibilidad de acceder a un instructivo para la utilizaci\'{o}n de Microsoft Word y Acrobat Professional, el cual est\'{a} disponible en la secci\'{o}n "Servicios", opci\'{o}n "Tr\'{a}mites" y enlace "Entrega de tesis".\\

% La redacci\'{o}n debe ser impersonal y gen\'{e}rica. La numeraci\'{o}n de las hojas sugiere que las p\'{a}ginas preliminares se realicen en n\'{u}meros romanos en may\'{u}scula y las dem\'{a}s en n\'{u}meros ar\'{a}bigos, en forma consecutiva a partir de la introducci\'{o}n que comenzar\'{a} con el n\'{u}mero 1. La cubierta y la portada no se numeran pero si se cuentan como p\'{a}ginas.\\

% Para trabajos muy extensos se recomienda publicar m\'{a}s de un volumen. Se debe tener en cuenta que algunas facultades tienen reglamentada la extensi\'{o}n m\'{a}xima de las tesis  o trabajo de investigaci\'{o}n; en caso que no sea as\'{\i}, se sugiere que el documento no supere 120 p\'{a}ginas.\\

% No se debe utilizar numeraci\'{o}n compuesta como 13A, 14B \'{o} 17 bis, entre otros, que indican superposici\'{o}n de texto en el documento. Para resaltar, puede usarse letra cursiva o negrilla. Los t\'{e}rminos de otras lenguas que aparezcan dentro del texto se escriben en cursiva.\\

\section{Planteamiento del Problema}
La tecnología se desarrolla continuamente mejorando la existente o simplemente dejando obsoleta
a la tecnología anterior. Este proceso es aún más acelerado en informática con la creación
continua de \textit{frameworks}, lenguajes e incluso paradigmas de programación. Tomando en cuenta
que el IoT es el conjunto de diversas tecnologías, entre las cuales se encuentra la informática,
no es extrañar que su desarrollo sea muy dinámico. En el mundo actual y en el futuro es muy
posible que los ingenieros se encuentren con situaciones en las que tengan que trabajar con el
Internet de las cosas. Es cierto que el los profesionales deben mantenerse actualizados ante la
mayor cantidad de novedades en la ciencia y en la industria que sea posible. Sin embargo, el aprendizaje de
nuevas tecnologías se dificulta con el pasar de los años y más aún cuando no existe un acercamiento
previo sobre el área de estudio.\\
Este problema no solo afecta al estudiante en su vida laboral sino también dentro de su desenvolvimiento
académico. Cuando un estudiante se encuentra en una situación en la que deba
aplicar cosas sobre las que tiene muy poca información o no disponga de referencias o puntos
de partida claros es muy posible que fracase en el primer intento o incluso en varios.
\subsection{Definición del problema}
El problema observado se encuentra en la ausencia de un acercamiento práctico al IoT mediante
laboratorios guiados y un sistema que coadyuve al mejor desenvolvimiento de los estudiantes.
Puesto que, no haber tenido dicho acercamiento previo la información y sobre todo herramientas
se encuentran incompletas y en consecuencia el estudiante deberá completarlas a contrarreloj
con el riesgo de fracasar en el intento. Esto se ve reflejado en la limitación que tendrá para
desarrollar proyectos con IoT. Pero también debido a que pierde la ventaja competitiva en el
mundo laboral relacionado al Internet de las cosas.
\section{Objetivos}
\subsection{Objetivo General}
Desarrollar un framework y un laboratorio interactivo para el desarrollo de aplicaciones IoT en
tiempo real.
\subsection{Objetivos específicos}
\begin{itemize}
	\item Desarrollar el sistema de comunicación en tiempo real para las aplicaciones y los dispositivos
tomando en cuenta la posibilidad de que existan múltiples canales de comunicación.
	\item Desarrollar el sistema soporte para las aplicaciones de los usuarios.
	\item Desarrollar una API que permita la utilización del sistema de comunicación en las aplicaciones
desarrolladas por los estudiantes.
	\item Implementar la infraestructura necesaria en el \textit{backend}.
	\item Desarrollar una API interna al sistema para la gestión de recursos consumidos por parte
de las aplicaciones.
	\item Desarrollar la interfaz gráfica que permitirá al usuario interactuar con el sistema.
	\item Seleccionar un dispositivo que cuente con el hardware necesario para conectarse a Internet
(\textit{Wi-Fi} o \textit{Ethernet}).
	\item Desarrollar una API que permita la comunicación con el dispositivo seleccionado.
	\item Redactar un laboratorio guiado que le permita al usuario desarrollar una aplicación.
	\item Desarrollar al menos una aplicación que exponga las características del sistema.
\end{itemize}
\section{Justificación}
El sistema propuesto será de gran utilidad en el aprendizaje del Internet de las cosas. Se trata
de un entorno que coadyuvará en el aprendizaje de los estudiantes y mejorará su posición para
enfrentarse a proyectos con IoT. En consecuencia el estudiante con conocimiento sobre IoT dispondría de una ventaja sobre sus pares que no lo tienen en la industria 4.0.
\section{Límites y Alcances}
\subsection{Límites}
\begin{itemize}
	\item El sistema no será totalmente compatible con navegadores para dispositivos móviles.
	\item La comunicación entre dispositivos y aplicaciones no será finalizada por el servidor por lo que el cliente deberá cerrar su conexión.
	\item No se desarrollaran controladores específicos para el hardware, sino que se trabajara sobre los existentes.
	\item No se dará soporte a distintos niveles de QoS (\textit{Quality of Service}).
	\item El sistema no proporcionará acceso a la base de datos del sistema desde las aplicaciones
o dispositivos.
	\item El \textit{backend} no podrá ser modificado por el usuario.
	\item No se dará soporte para múltiples protocolos de comunicación entre dispositivos.
\end{itemize}
\subsection{Alcances}

\begin{itemize}
	\item El sistema permitirá al usuario administrar los dispositivos que se conectan a sus aplicaciones.
	\item Será posible enviar y recibir mensajes a las aplicaciones y dispositivos mediante la interfaz gráfica.
	\item El sistema podrá controlar el nivel de acceso de sus aplicaciones según lo especifique el usuario.
	\item El sistema será capaz de gestionar múltiples aplicaciones de forma simultánea.
	\item El sistema proporcionará un medio de almacenamiento accesible al usuario desde la interfaz gráfica.
	\item El sistema funcionará con aplicaciones \textit{WebAssembly} como interfaz gráfica para las aplicaciones de los usuarios.
    \item Se elaborará un manual descriptivo del sistema basado en ejemplos.
    \item El manual contendrá una guía paso a paso para la elaboración de una aplicación demostrativa.
    \item El sistema funcionará en los navegadores \textit{Google Chrome}, \textit{Microsoft Edge}, \textit{Mozilla Firefox}
y \textit{Safari} en sus versiones de escritorio.
    \item El hardware seleccionado tendrá como mínimo una interfaz de red para conectarse a Internet,
ya sea \textit{Wi-Fi} o \textit{Ethernet}.
\end{itemize}
