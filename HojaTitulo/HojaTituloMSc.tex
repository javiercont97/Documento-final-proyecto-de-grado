%\newpage
%\setcounter{page}{1}
\begin{center}
\begin{figure}
\centering%
\textsc{\large UNIVERSIDAD CAT\'OLICA BOLIVIANA ``SAN PABLO'' }\\[0.3cm] % Name of your university/college
\textsc{\large UNIDAD ACAD\'EMICA REGIONAL LA PAZ}\\[0.3cm] %
\textsc{\large FACULTAD DE INGENIER\'IA}\\[0.3cm]
\textsc{\normalsize  CARRERA DE INGENIER\'IA MECATR\'ONICA}\\[0.1cm]
\epsfig{file=HojaTitulo/UCBSinFondo.png,scale=0.75}%
\end{figure}
\textbf{\large
LABORATORIO PARA EL DESARROLLO DE APLICACIONES IOT EN TIEMPO REAL PARA ENTORNOS ACAD\'EMICOS}\\[0.5cm]

\thispagestyle{empty} \vspace*{0.01cm} \textbf{\large Proyecto de grado presentado para la obtenci\'on del Grado de Ingenier\'ia Mecatr\'onica}\\[0.65cm]

\thispagestyle{empty} \vspace*{0.01cm} \normalsize Por: JAVIER MAURICIO CONTRERAS GUZMAN \\[0.8cm]

\thispagestyle{empty} \vspace*{0.01cm} \normalsize Tutor: MIGUEL ANGEL CLAVIJO QUISPE\\[1.5cm]

\vspace*{0.01cm} \normalsize La Paz-Bolivia\\[0.25cm]
\vspace*{0.01cm} \normalsize Septiembre, 2020
\end{center}

\newpage{\pagestyle{empty}\cleardoublepage}

\newpage

\newpage{\pagestyle{empty}\cleardoublepage}

\newpage
\thispagestyle{empty} \textbf{}\normalsize
\\\\\\%
\textbf{DEDICATORIA}\\[4.0cm]

\begin{flushright}
\begin{minipage}{8cm}
    \noindent
        \small
        \lipsum[0-1]
        % La dedicatoria es opcional y cada autor podr\'a determinar la distribuci\'on del texto en la p\'{a}gina, se sugiere esta presentaci\'on. En ella el autor dedica su trabajo en forma especial a personas y/o entidades.\\[1.0cm]\\
        % Por ejemplo:\\[1.0cm]
        % A mis padres\\[1.0cm]\\
        % o\\[1.0cm]
        % La preocupaci\'on por el hombre y su destino siempre debe ser el
        % inter\'es primordial de todo esfuerzo t\'ecnico. Nunca olvides esto
        % entre tus diagramas y ecuaciones.\\\\
        % Albert Einstein\\
\end{minipage}
\end{flushright}

%\newpage{\pagestyle{empty}\cleardoublepage}

\newpage
\thispagestyle{empty} \textbf{}\normalsize
\\\\\\%
\textbf{AGRADECIMIENTOS}
%\addcontentsline{toc}{chapter}{\numberline{}Agradecimientos}\\\\
\\\\
\lipsum[0-2]
% Esta secci\'{o}n es opcional, en ella el autor agradece a las personas o instituciones que colaboraron en la realizaci\'{o}n de la tesis  o trabajo de investigaci\'{o}n. Si se incluye esta secci\'{o}n, deben aparecer los nombres completos, los cargos y su aporte al documento.\\

\newpage{\pagestyle{empty}\cleardoublepage}

\newpage
\textbf{\LARGE Resumen}\\\\
%\addcontentsline{toc}{chapter}{\numberline{}Resumen}
\\\\
\lipsum[0-1]
% El resumen es una presentaci\'{o}n abreviada y precisa (la NTC 1486 de 2008 recomienda revisar la norma ISO 214 de 1976). Se debe usar una extensi\'{o}n m\'{a}xima de 15 renglones. Se recomienda que este resumen sea anal\'{\i}tico, es decir, que sea completo, con informaci\'{o}n cuantitativa y cualitativa, generalmente incluyendo los siguientes aspectos: objetivos, dise\~{n}o, lugar y circunstancias, objetivo del estudio, intervenci\'{o}n, mediciones y principales resultados, y conclusiones. Al final del resumen se deben usar palabras claves tomadas del texto (m\'{\i}nimo 3 y m\'{a}ximo 7 palabras), las cuales permiten la recuperaci\'{o}n de la informaci\'{o}n.\\

\textbf{\small \textit{\lipsum[0-1][10]}}

% \textbf{\small \textit{Palabras clave: (m\'{a}ximo 10 palabras, preferiblemente seleccionadas de las listas internacionales que permitan el indizado cruzado)}}.\\

\textbf{\small L\'inea de investigaci\'on:} \small \lipsum[0-1][10]

% \textbf{\small L\'inea de investigaci\'on:} \small (m\'aximo 1 o 2  rengl\'ones en que se establezca la  l\'inea de investigaci\'on a la que pertenece el proyecto de grado).\\


\newpage
\textbf{\LARGE Abstract}\\\\
\lipsum[0-1]
% Es el mismo resumen pero traducido al ingl\'{e}s. Se debe usar una extensi\'{o}n m\'{a}xima de 12 renglones. Al final del Abstract se deben traducir las anteriores palabras claves tomadas del texto (m\'{\i}nimo 3 y m\'{a}ximo 7 palabras), llamadas keywords. Es posible incluir el resumen en otro idioma diferente al espa\~{n}ol o al ingl\'{e}s, si se considera como importante dentro del tema tratado en la investigaci\'{o}n, por ejemplo: un trabajo dedicado a problemas ling\"{u}\'{\i}sticos del mandar\'{\i}n seguramente estar\'{\i}a mejor con un resumen en mandar\'{\i}n.\\[2.0cm]

\textbf{\small \textit{\lipsum[0-1][10]}}

% \textbf{\small \textit {Keywords: palabras clave en ingl\'es(m\'aximo 10 palabras, preferiblemente seleccionadas de las listas internacionales que permitan el indizado cruzado)}}\\

\textbf{\small L\'inea de investigaci\'on:} \small \lipsum[0-1][10]

% \textbf{\small Research area:} \small texto en ingl\'es (m\'aximo 1 o 2  rengl\'ones en que se establezca la  l\'inea de investigaci\'on a la que pertenece el proyecto de grado).\\