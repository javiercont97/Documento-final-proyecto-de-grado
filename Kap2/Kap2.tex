\chapter{Marco Te\'orico}
% Los cap\'{\i}tulos son las principales divisiones del documento. En estos, se desarrolla el tema del documento. Cada cap\'{\i}tulo debe corresponder a uno de los temas o aspectos tratados en el documento y por tanto debe llevar un t\'{\i}tulo que indique el contenido del cap\'{\i}tulo.\\

% Los t\'{\i}tulos de los cap\'{\i}tulos deben ser concertados entre el alumno y el director de la tesis  o trabajo de investigaci\'{o}n, teniendo en cuenta los lineamientos que cada unidad acad\'{e}mica brinda. As\'{\i} por ejemplo, en algunas facultades se especifica que cada cap\'{\i}tulo debe corresponder a un art\'{\i}culo cient\'{\i}fico, de tal manera que se pueda publicar posteriormente en una revista.\\

\lipsum[0-2]

\section{Estado del Arte}
% Es evidenete que esta secci\'on trata de una extensa revisi\'on bibliogr\'afica, por lo que se deber\'a referenciar una cantidad importante de papers. 

% Existen varias normas para la citaci\'{o}n bibliogr\'{a}fica. Algunas \'{a}reas del conocimiento prefieren normas espec\'{\i}ficas para citar las referencias bibliogr\'{a}ficas en el texto y escribir la lista de bibliograf\'{\i}a al final de los documentos. Esta plantilla brinda la libertad para que el autor de la tesis  o trabajo de investigaci\'{o}n utilice la norma bibliogr\'{a}fica com\'{u}n para su disciplina. Sin embargo, se solicita que la norma seleccionada se utilice con rigurosidad, sin olvidar referenciar "todos" los elementos tomados de otras fuentes (referencias bibliogr\'{a}ficas, patentes consultadas, software empleado en el manuscrito, en el tratamiento a los datos y resultados del trabajo, consultas a personas (expertos o p\'{u}blico general), entre otros).\\

% Existen algunos ejemplos para la citaci\'{o}n bibliogr\'{a}fica, por ejemplo, Microsoft Word (versiones posteriores al 2006), en el  men\'{u} de referencias, se cuenta con la opci\'{o}n de insertar citas bibliogr\'{a}ficas utilizando la norma APA (American Psychological Association) u otras normas y con la ayuda para construir autom\'{a}ticamente la lista al final del documento. De la misma manera, existen administradores bibliogr\'{a}ficos compatibles con Microsoft Word como Zotero, End Note y el Reference Manager,  disponibles a trav\'{e}s del Sistema Nacional de Bibliotecas (SINAB) de la Universidad Nacional de Colombia\footnote{Ver:www.sinab.unal.edu.co } secci\'{o}n "Recursos bibliogr\'{a}ficos" opci\'{o}n "Herramientas Bibliogr\'{a}ficas. A continuaci\'{o}n se muestra un ejemplo de una de las formas m\'{a}s usadas para las citaciones bibliogr\'{a}ficas.\\

% Citaci\'{o}n individual:\cite{AG01}.\\
% Citaci\'{o}n simult\'{a}nea de varios autores:
% \cite{AG12,AG52,AG70,AG08a,AG09a,AG36a,AG01i}.\\

% Por lo general, las referencias bibliogr\'{a}ficas correspondientes a los anteriores n\'{u}meros, se listan al final del documento en orden de aparici\'{o}n o en orden alfab\'{e}tico. Otras normas de citaci\'{o}n incluyen el apellido del autor y el a\~{n}o de la referencia, por ejemplo: 1) "...\'{e}nfasis en elementos ligados al \'{a}mbito ingenieril que se enfocan en el manejo de datos e informaci\'{o}n estructurada y que seg\'{u}n Kostoff (1997) ha atra\'{\i}do la atenci\'{o}n de investigadores dado el advenimiento de TIC...", 2) "...Dicha afirmaci\'{o}n coincide con los planteamientos de Snarch (1998), citado por Castellanos (2007), quien comenta que el manejo..." y 3) "...el futuro del sistema para argumentar los procesos de toma de decisiones y el desarrollo de ideas innovadoras (Nosella \textsl{et al}., 2008)...".\\

\lipsum[0-3]

\subsection{Enfoque 1}
% De la cuarta subdivisi\'{o}n en adelante, cada nueva divisi\'{o}n o \'{\i}tem puede ser se\~{n}alada con vi\~{n}etas, conservando el mismo estilo de \'{e}sta, a lo largo de todo el documento.\\

% Las subdivisiones, las vi\~{n}etas y sus textos acompa\~{n}antes deben presentarse sin sangr\'{\i}a y justificados.\\
\lipsum[0-1]

\subsection{Enfoque 2}
% De la cuarta subdivisi\'{o}n en adelante, cada nueva divisi\'{o}n o \'{\i}tem puede ser se\~{n}alada con vi\~{n}etas, conservando el mismo estilo de \'{e}sta, a lo largo de todo el documento.\\
\lipsum[0-1]
\subsection{Enfoque n}
% De la cuarta subdivisi\'{o}n en adelante, cada nueva divisi\'{o}n o \'{\i}tem puede ser se\~{n}alada con vi\~{n}etas, conservando el mismo estilo de \'{e}sta, a lo largo de todo el documento.\\
\lipsum[0-1]

\subsection{Discusi\'on}
% De la cuarta subdivisi\'{o}n en adelante, cada nueva divisi\'{o}n o \'{\i}tem puede ser se\~{n}alada con vi\~{n}etas, conservando el mismo estilo de \'{e}sta, a lo largo de todo el documento.\\

% Es recomendable incluir una tabla resumen de comparaci\'on entre las diferentes soluciones.

% Para la edici\'{o}n de tablas, cada columna debe llevar su t\'{\i}tulo; la primera palabra se debe escribir con may\'{u}scula inicial y preferiblemente sin abreviaturas. En las tablas y cuadros, los t\'{\i}tulos y datos se deben ubicar entre l\'{\i}neas horizontales y verticales cerradas (como se realiza en esta plantilla).\\

% La numeraci\'{o}n de las tablas se realiza de la misma manera que las figuras o ilustraciones, a lo largo de todo el texto. Deben llevar un t\'{\i}tulo breve, que concreta el contenido de la tabla; \'{e}ste se debe escribir en la parte superior de la misma. Para la presentaci\'{o}n de cuadros, se deben seguir las indicaciones dadas para las tablas.\\

% Un ejemplo para la presentaci\'{o}n y citaci\'{o}n de tablas (citaci\'{o}n indirecta), se presenta a continuaci\'{o}n:\\
\lipsum[0-2]

% \begin{center}
% 	\begin{threeparttable}
% 		\centering%
% 		\caption{Participaci\'{o}n de las energ\'{\i}as renovables en el suministro
% 			total de energ\'{\i}a primaria \cite{AG02i}.}\label{EMundo1}
% 		\begin{tabular}{|l|c|c|}\hline
% 			&\multicolumn{2}{c|}{Participaci\'{o}n en el suministro de energ\'{\i}a primaria /\% (Mtoe)\;$\tnote{1}$}\\\cline{2-3}%
% 			\arr{Region}&Energ\'{\i}as renovables &Participaci\'{o}n de la biomasa\\\hline%
% 			Latinoam\'{e}rica&28,9 (140)&62,4 (87,4)\\\hline%
% 			\:Colombia&27,7 (7,6)&54,4 (4,1)\\\hline%
% 			Alemania&3,8 (13,2)&65,8 (8,7)\\\hline%
% 			Mundial&13,1 (1404,0)&79,4 (1114,8)\\\hline
% 		\end{tabular}
% 		\begin{tablenotes}
% 			\item[1] \footnotesize{1 kg oe=10000 kcal=41,868 MJ}
% 		\end{tablenotes}
% 	\end{threeparttable}
% \end{center}

\section{Fundamentos Te\'oricos}
\lipsum[0-1]
% \subsection{T\'ecnica/concepto/Algoritmo 1}
% Recuerde incluir la ecuaciones debidas como se muestra a continuaci\'on:
% 	\begin{equation}\label{fuerza}
% 	F = ma
% 	\end{equation}
% Donde: \\ 
% \(F\) es fuerza\\
% \(m\) es masa\\
% \(a\) es aceleraci\'on\\
% La ecuaci\'on  \ref{fuerza} es utilizada para determinar la .... 

\subsection{T\'ecnica/concepto/Algoritmo 2}
\subsection{T\'ecnica/concepto/Algoritmo 3}
\subsection{T\'ecnica/concepto/Algoritmo n}

